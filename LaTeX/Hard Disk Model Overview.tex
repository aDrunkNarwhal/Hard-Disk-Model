\documentclass[a4paper,11pt]{article}
\usepackage[T1]{fontenc}
\usepackage[utf8]{inputenc}
\usepackage{lmodern}
\usepackage{fullpage}

\title{Phase Transitions in the Hard Spheres Model}
\author{Matt Peterson \thanks{University of New Mexico for Computer Science, mpeterson@unm.edu. Sandia National Laboratories, mgpeter@sandia.gov} \and Thomas Hayes \thanks{University of New Mexico for Computer Science, hayes@cs.unm.edu}}

\begin{document}

\maketitle
%\tableofcontents

\section*{Motivation}

Why does a collection of water molecules suddenly freeze or boil when we slightly change it's temperature?  From the perspective of individual water molecules, very little seems to change from 99$^\circ$C to 101$^\circ$C.  But macroscopically, we see a huge qualitative difference.  More generally, even very simple mathematical models of collections of particles are observed to undergo dramatic ``phase transitions'' as a controlling parameter, such as temperature or density, is varied.

The most natural transition to study is from a gaseous phase, defined as one in which there is no long-distance correlation between the positions of particles, and a more crystalline phase in which there is.  From a physics point of view, one looks for a critical point in a controlling parameter.  With density $\rho$, one tries to find the critical density $\rho_c$ where $\rho < \rho_c$ the molecules are in a gaseous configuration, and $\rho > \rho_c$ they are in a crystalline configuration.

A seemingly different way to view the problem is through the eyes of a computer scientist.  If one wants a randomly distributed configuration it is important to know the convergence rate.  It is believed that there is some $\rho_c$ where $\rho < \rho_c$ the convergence rate scales polynomially, and $\rho > \rho_c$ it scales exponentially.

There is some evidence that the answers to these two questions are related.  Empirical evidence has shown that the convergence rate is faster when the configuration is gaseous and slower when it is crystalline.  But the values for these convergence rates are not known.

\section*{Proposed Work}

%TODO: Talk about model. Talk about single-sphere global-move dynamics. Edit modified Hamming metric. Talk about even chain dynamics. Explain why we think this will work better in crystalline configurations.


Numerical results (e.g. \cite{Mak}\cite{Piasecki}) show that the critical density for a phase transition to a solid configuration is $\rho_c \approx 0.7$. Our intuition tells us that the bound on critical density for mixing time is $\rho_c \le 0.7$. 

One of the easiest ways is to pick up a sphere and attempt to move it to a random location within the box.  This process is referred to as \textit{single-sphere global-move dynamics}.  More precisely a sphere $i$ is pick uniformly at random and a location $x$ in the box is also picked uniformly at random.  If the sphere $i$ can be placed at location $x$ without overlapping another sphere then we do it, otherwise we leave sphere $i$ in its original location.  Using this dynamic and a path coupling argument the result yields $\rho_c \le 1/8 = 0.125$ \cite{Kannan} for when the box is two dimensions.

Hayes and Moore \cite{Hayes} later improved the bound to $\rho_c \le 0.154483..$.  To accomplish this they used the same single-sphere global-move dynamics and path coupling arguments, but modified the Hamming metric.  The normal Hamming metric is 0 if the spheres are in the same location in both chains, and 1 if the spheres are in different locations.  Hayes and Moore designed a Hamming metric that depends continuously on the difference between the two positions.  More importantly it goes to zero continuously as the two positions coincide.

Instead of using single-sphere global-move dynamics one could use a \textit{event chain dynamics}.  The idea behind this is to pick a sphere and direction uniformly at random.  The sphere attempts to move in the picked direction a distance $\epsilon$.  If the sphere bumps into another sphere while moving, the picked sphere stops moving and the bumped into sphere begins moving is the same direction for the remainder of the distance.  If this new moving sphere bumps into another sphere, the process is repeated.

The intuition that this dynamic might perform better is that it is used in practice.  By empirical observations this approach appears to mix well.  The hope is that this approach may lead to a bound closer to $\rho_c \le 0.7$.

There is a variation of this model where spheres are allowed to enter and leave the box (i.e. toroid).  This adds an extra variable $\lambda$ to the system which controls to probability of a sphere entering or leaving the system.  Since $N$ is not constant, neither is the density $\rho$.  A question that could be asked about this model would be how $\rho$ and $\lambda$ are related to each other.  Given some fixed $\lambda$ value, what would $\rho$ converge to after a sufficient amount of time steps? Or is there some critical range for $\lambda$ where the density never converges?

\begin{thebibliography}{9}

%\bibitem{Bernard}
%  Etienne Bernard, Werner Krauth, and David Wilson.
%  \emph{Event-chain monte carlo algorithms for hard-sphere systems.}
%  Physical Review E,
%  80(5):056704, 2009.

\bibitem{Mak}
  Chi Mak. 
  \emph{Large-scale simulations of the two-dimensional melting of hard disks}.
  Phys. Rev. E,
  73(6):065104, 2006.

\bibitem{Piasecki}
  Jaroslaw Piasecki, Piotr Szymczak, and John Kozak.
  \emph{Prediction of a structural transition in the hard disk fluid}.
  The Journal of chemical physics,
  133:164507, 2010.

\bibitem{Kannan}
  Ravi Kannan, Michael Mahoney, and Ravi Montenegro.
  \emph{Rapid mixing of several Markov chains for a hard-core model}.
  In Proc. 14th Intl. Symp. on Algorithms and Computation (ISAAC),
  pages 663–675, 2003.

\bibitem{Hayes}
  Thomas Hayes and Cristopher Moore.
  \emph{Lower Bounds on the Critical Density in the Hard Disk Model via Optimized Metrics}.
  arXiv preprint,
  arXiv:1407.1930 (2014).

\end{thebibliography}

\end{document}
